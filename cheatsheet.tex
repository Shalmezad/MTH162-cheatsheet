\documentclass[10pt,landscape]{article}
\usepackage{multicol}
\usepackage[landscape]{geometry}

% Remove fluff
\pagestyle{empty}
\setcounter{secnumdepth}{0}

%\setlength{\parskip}{0.1em}
\geometry{top=.5in,left=.5in,right=.5in,bottom=.5in}

\begin{document}
	
	\raggedright
	\footnotesize
	\begin{multicols}{4}
\section{Useful Stuff}

\paragraph{Completing the square}

$ax^2 + bx + c = 0$

$a(x+d)^2 + e = 0$

$d = \frac{b}{2a}$

$e = c - \frac{b^2}{4a}$

\subsection{Trig Functions}

$sin^2(x) = \frac{1-cos(2x)}{2}$

$cos^2(x) = \frac{1+cos(2x)}{2}$

$\frac{d}{dx}tan(x) = sec^2(x)$

$\frac{d}{dx}sec(x) = sec(x)tan(x)$

$\frac{d}{dx}tan^{-1}(x) = \frac{1}{1+x^2}$

$\int sec(x) = ln|sec(x)+tan(x)|$


\section{Area \& Volume}

\paragraph{Volume by slicing}

A(x) = area of cross section at x,

$V(S) = \int_{a}^{b}A(x)dx$

\paragraph{Volume by disk}

If f(x) rotated around x-axis,

$V(S) = \int_{a}^{b}\pi [f(x)]^2dx$

\paragraph{Volume by cylindrical shells}

If f(x) rotated around y-axis,

$V(S) = \int_{a}^{b}(2\pi x f(x))dx$

\section{Applications}

\paragraph{Arc Length}

$\int_{a}^{b} \sqrt{1+[f'(x)]^2}dx$

Note use of derivative!

\paragraph{Surface Area of Revolution}

Revolve f(x) around x axis,

$SA(x) = \int_{a}^{b}(2 \pi f(x) \sqrt{1+[f'(x)]^2})dx$

\paragraph{Mass-density for 1-d object}

If p(x) linear density for given x,

$m = \int_{a}^{b}p(x)dx$

\paragraph{Mass-density for circular object}

If p(x) radial density for given x, and radius = r,

$m = \int_{0}^{r}2 \pi x p(x)dx$

\paragraph{Work done}

If F(X) = force at point x,

$W = \int_{a}^{b} F(x) dx$

*Recall constant force yields $F*d$

\subsection{Hyperbolic Functions}

\begin{tabular}{l l}
$f(x)$ & $\frac{d}{dx}f(x)$ \\
\hline
$sinh(x)$ & $cosh(x)$ \\
$cosh(x)$ & $sinh(x)$ \\
$tanh(x)$ & $sec^2(x)$ \\
$coth(x)$ & $-csch^2(x)$ \\
$sech(x)$ & $-sech(x)tanh(x)$ \\
$csch(x)$ & $-csch(x)coth(x)$ \\
\hline
\end{tabular}

\section{Integration Techniques}

\paragraph{Int by parts} 
$\int u dv = uv - \int v du$

Pick u using LIATE (log, inv trig, alg, trig, exp)

\hrulefill

\paragraph{$ \int cos^j(x)sin^k(x)dx $}
\subparagraph{If k odd}
keep 1 $sin(x)$, convert rest using $sin^2x = 1-cos^2x$. u-sub with $u=cos(x)$.
\subparagraph{If j odd}
keep 1 $cos(x)$, convert rest using $cos^2x = 1-sin^2x$. u-sub with $u=sin(x)$.
\subparagraph{If both even}
use $sin^2x=\frac{1-cos(2x)}{2}$.

\paragraph{$ \int tan^k(x)sec^j(x)dx $}
\subparagraph{If j even and $\ge 2$}
keep $sec^2(x)$, convert rest using $sec^2x=tan^2x+1$. u-sub with $u=tanx$.
\subparagraph{If k odd, $j \ge 1$}
keep $sec(x)tan(x)$, convert rest using $tan^2x=sec^2x$. u-sub with $u=secx$.
\subparagraph{If k odd, $k \ge 3$ and $j=0$}
turn one $tan^2x$ into $sec^2x-1$. Repeat process.
If k even, j odd, use $tan^2x=sec^2x-1$ to turn $tan^kx$ to $secx$.

\paragraph{Reductions}
\subparagraph{}
$ \int sec^nx dx = \frac{1}{n-1}sec^{n-2}xtanx+\frac{n-2}{n-1}\int sec^{n-2}x dx$
\subparagraph{}
$ \int tan^nx dx = \frac{1}{n-1}tan^{n-1}x-\int tan^{n-2}x dx$

\paragraph{Trig subs}
Don't forget to change dx to d$\theta$

\begin{tabular}{l l}
$a^2-x^2$ & Use $x=asin\theta$ \\
$a^2+x^2$ & Use $x=atan\theta$ \\
$x^2-a^2$ & Use $x=asec\theta$
\end{tabular}

\paragraph{Arclength $ax^2$}
$\frac{x\sqrt{1+4a^2x^2}}{2} + \frac{ln(|\sqrt{1+4a^2x^2}+2ax|)}{4a}$

Note: if evaluating over interval [a,b], if a=0, just plug in b

\paragraph{Reduction $\int \frac{1}{(ax^2+b)^n}$}
$\frac{2n-3}{2b(n-1)} \int \frac{1}{(ax^2+b)^{n-1}}dx + \frac{x}{2b(n-1)(ax^2+b)^{n-1}}$
		
	\end{multicols}
\end{document}